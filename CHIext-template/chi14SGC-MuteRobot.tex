\documentclass{chi-ext}
% Please be sure that you have the dependencies (i.e., additional LaTeX packages) to compile this example.
% See http://personales.upv.es/luileito/chiext/

%% EXAMPLE BEGIN -- HOW TO OVERRIDE THE DEFAULT COPYRIGHT STRIP -- (July 22, 2013 - Paul Baumann)

%\copyrightinfo{Permission to make digital or hard copies of all or part of this work for personal or classroom use is granted without fee provided that copies are not made or distributed for profit or commercial advantage and that copies bear this notice and the full citation on the first page. Copyrights for components of this work owned by others than ACM must be honored. Abstracting with credit is permitted. To copy otherwise, or republish, to post on servers or to redistribute to lists, requires prior specific permission and/or a fee. Request permissions from permissions@acm.org. \\
% {\emph{CHI'14}}, April 26--May 1, 2014, Toronto, Canada. \\
% Copyright \copyright~2014 ACM ISBN/14/04...\$15.00.}

\copyrightinfo{\scriptsize Permission to make digital or hard copies of part or all of this work for personal or classroom use is granted without fee provided that copies are not made or distributed for profit or commercial advantage and that copies bear this notice and the full citation on the first page. Copyrights for third-party components of this work must be honored. For all other uses, contact the owner/author(s). Copyright is held by the author/owner(s). \\
{\emph{CHI 2014}}, April 26--May 1, 2014, Toronto, Ontario, Canada. \\
ACM 978-1-4503-2474-8/14/04. \\
http://dx.doi.org/10.1145/2559206.2580099 }
\clubpenalty=10000 
\widowpenalty = 10000

% DOI string from ACM form confirmation}
%% EXAMPLE END -- HOW TO OVERRIDE THE DEFAULT COPYRIGHT STRIP -- (July 22, 2013 - Paul Baumann)

%\title{CHI \LaTeX\ Ext. Abstracts Template}
\title{Mute Robot - Cooperative Gameplay through Body Language Communication}

\numberofauthors{4}
% Notice how author names are alternately typesetted to appear ordered in 2-column format;
% i.e., the first 4 autors on the first column and the other 4 auhors on the second column.
% Actually, it's up to you to strictly adhere to this author notation.
\author{
  \alignauthor{
  	\textbf{Chun-Yen Hsu}\\
  	\affaddr{Department of Computer Science and Information Engineering}\\
  	\affaddr{National Taiwan University, Taipei, Taiwan}\\
  	%\affaddr{Authortown, PA 54321 USA}\\
  	\email{hcythomas0125@gmail.com}
  }\alignauthor{
  	\textbf{Han-Yu Wang}\\
  	\affaddr{Department of Networking and Multimedia}\\
  	\affaddr{National Taiwan University, Taipei, Taiwan}\\
  	%\affaddr{Authortown, PA 54321 USA}\\
  	\email{huw12313212@gmail.com}
  }
  \vfil
  \alignauthor{
  	\textbf{Ying-Chao Tung}\\
  	\affaddr{Department of Computer Science and Information Engineering}\\
  	\affaddr{National Taiwan University, Taipei, Taiwan}\\
  	%\affaddr{Authortown, PA 54321 USA}\\
  	\email{tony61507@gmail.com}
  }%\alignauthor{
  	%\textbf{Fifth Author}\\
  	%\affaddr{AuthorCo, Inc.}\\
  	%\affaddr{123 Author Ave.}\\
  	%\affaddr{Authortown, PA 54321 USA}\\
  	%\email{author6@anotherco.com}
  %}
  \vfil
  \alignauthor{
  	\textbf{Wei-Han Wang}\\
  	\affaddr{Department of Networking and Multimedia}\\
  	\affaddr{National Taiwan University, Taipei, Taiwan}\\
  	%\affaddr{Authortown, PA 54321 USA}\\
  	\email{wangweihang5566@gmail.com}
  }%\alignauthor{
  	%\textbf{Sixth Author}\\
  	%\affaddr{AuthorCo, Inc.}\\
  	%\affaddr{123 Author Ave.}\\
  	%\affaddr{Authortown, PA 54321 USA}\\
  	%\email{author7@anotherco.com}
  %}
}

% Paper metadata (use plain text, for PDF inclusion and later re-using, if desired)
\def\plaintitle{CHI LaTeX Extended Abstracts Template}
\def\plainauthor{Luis A. Leiva}
\def\plainkeywords{Game; Video games; Game design; Body language; Kinect}
\def\plaingeneralterms{Documentation, Standardization}

\hypersetup{
  % Your metadata go here
  pdftitle={\plaintitle},
  pdfauthor={\plainauthor},  
  pdfkeywords={\plainkeywords},
  pdfsubject={\plaingeneralterms},
  % Quick access to color overriding:
  %citecolor=black,
  %linkcolor=black,
  %menucolor=black,
  %urlcolor=black,
}

\usepackage{graphicx}   % for EPS use the graphics package instead
\usepackage{balance}    % useful for balancing the last columns
\usepackage{bibspacing} % save vertical space in references


\begin{document}

\maketitle

\begin{abstract}
%Speech ability, a general habit becoming nature for us, plays an important role in our daily life. Besides speech, using body language is an evident and feasible way to communicate between two people.
Body language is an expressive form of communication that transcends language barriers, and can range from subtle to outrageous. 
We have designed Mute Robot, a game in which 2 players must cooperate to solve a series of puzzle challenges by communicating through body language only.
Kinect devices are used to capture players' posture and movement, which are then shared between two partners who are playing at two different physical locations. Mute Robot is designed to connect people across the globe who otherwise would not be able to communicate via a common language.
%We encourage players to communicate with body language, with an emphasis on cooperation to facilitate immersion with each other.
%In our 16-person user study, all participants thought it was interesting to communicate with body language 
Our 16-person user study showed that body language is practical and entertaining, with 2/3 of the players reported that they could understand the other player's body language well.
%Players need to cooperate with each other to pass challenges by using body language communication.
\end{abstract}

%\begin{abstract}
%In this sample we describe the formatting requirements for various %SIGCHI related submissions 
%and offer recommendations on writing for the worldwide SIGCHI %readership. 
%%Do not change the page size or page settings.
%Please review this document even if you have submitted to SIGCHI %conferences before, 
%some format details have changed relative to previous years.
%\end{abstract}

%[Section] Author Keywords
\keywords{\plainkeywords}
%\textcolor{red}{Mandatory section to be included in your final version.}

%[Section] ACM Classification Keywords
\category{K.8.0}{Personal Computing}{General \it{Games}}. 
%See \cite{ACMCCS} 
%See: \url{http://www.acm.org/about/class/1998/} 
%for help using the ACM Classification system.
%\textcolor{red}{Mandatory section to be included in your final version.}

%[Section] General Terms
%\terms{\plaingeneralterms}
%\textcolor{red}{Optional section to be included in your final version.}

% =============================================================================
\section{Introduction}
% =============================================================================

%Before any development of languages, it was feasible that people use body language to build a connection between each other.
%Furthermore, it is a common solution for human using body langauge to communicate with people from different regions.
Body language is proven by previous studies\cite{BL1,BL2} that it carries much of the emotive content. It may provide clues as to the attitude or state of mind of a person. For example, it may indicate aggression, attentiveness, boredom, a relaxed state, pleasure, amusement, and intoxication\cite{BLWiki}.
Body language is also the fundamental form of communication. 
It is how we communicate with babies who have not developed language skills, and how we communicate with people whom we do not share a common language with.

%builds connection between people without speech, and is common for people to communicate with people from different regions.
%With the appearance of motion capture devices, the diversity of gameplay increases dramatically.
With the recent advancement in motion capture devices, such as the Microsoft Kinect\cite{Kinect}, players are now able to play games with body motion and gestures.
However, most games have focused on using body motion as a more natural input to replace controllers.
Also, multiplayer games have focused on party settings with co-located players, rather than connecting people across the globe.
In our work, we aim to develop a cross-language game by leveraging subtlety and fun in body languages. Even when players do not share a common language, they can still connect with each other through cooperative gameplay.

%[Mike origin]
%%In order to apply better user experience and reduce the hardware %device limitation, 
%We have designed {\it Mute Robot}, an interactive game using Kinect to capture players' body movement and posture and using them to control their game avatars. 
%%Body languages composed of postures allow players to communicate with each other without speech.
%Mute Robot is a cooperative platformer game with two players connected over the Internet.
%By mapping players' motions to avatars in the game, Mute Robot attempts to create an innovative gameplay by letting players communicate with each other only using body language.
%%Without any other way of communication
%With only body language, players will build their own conversation pattern with body language during the game. 
%%at the beginning.
%[]


%After finishing building their own connection between each other, players start to solve a serial of puzzles with their own communication patterns.
%For example, in the first level, there is a door locked on the right side, players have to tread three buttons on the ground with the correct sequence to unlock the door.(fig. 2)
%However, only the upper side player know the sequence, the underside player can trample those buttons, so they need to pass the right pattern with the communication method they built before.

%The focus on Mute Robot lies on this cooperative-without-speech desgin for the core gameplay.
%With an eye on making all players experience both acting an puzzle-solving hints information sender and a receiver, we switch password stages between the upper side and the underside at different levels.
%Morevoer, Mute Robot desires to explore the limitation of communicating without normal methods, so the game require the password sender acting more and more difficult objects with body language.
%For example, in the second stage, there are a circle and a x on the first wheel and then three numbers on the second one. It increases the difficulty on the communication between the two players.(fig. 1)
%Thus, players need to evaluate their own connection modus stronger and stronger to deal with the increasing difficulty in the game. 

Mute Robot shows a novel integration between current cooperative gameplay and body language. Our 16-person user study results indicate that body language based cooperative gameplay is practical and entertaining. 
Compared to previous game designs that used keyboards and mice as input devices to move the avatars (e.g. Way\cite{Way}), Mute Robot makes use of body language which is significantly more expressive and captures the nuances in communication. 
We hope Mute Robot inspires future exploration of this relatively unexplored area of game design.  
%Mute Robot shows a new integration between current cooperative gameplay and body language. After our user study, the results also indicate that body language based manner is utility to increase the entertainment and cooperation between players. In addtion, it constructs a new innovative gameplay style from time immemorial. Mute Robot provides a path towards future exploration of this relatively unexplored field of game design with capacious development space.

% =============================================================================
\section{Cooperative Game Design Patterns}
% =============================================================================
There were some related works done by past researchers. Such as Zagal et al, analyzing cooperative game design patterns through board games\cite{CGDP1}; Bjork and Holopainen\cite{CGDP2}, supporting plenty of Game Design Patterns, inclusive of cooperative and social interaction patterns. In addition, Rocha et al\cite{CGDP3} provided several game design frameworks, and Magy et al\cite{CGDP4} integrated all and proposed methods to evaluate quality about Game Design Patterns. From experience mentioned above, the design of Mute Robot adopts the following design patterns: \textsl{Complementarity}, \textsl{Abilities that can only be used on another player}, \textsl{Shared Goals}, \textsl{Shared Puzzles}. 
 

 \begin{figure}
  \centering
  \includegraphics[width=\linewidth]{figures/Figure1.jpg}
  \caption{The asymmetric puzzle game design in one of Mute Robot's stages.}
  \label{fig:Figure1}
\end{figure}

  
% =============================================================================
\section{Design and Development}
% =============================================================================
Mute Robot is a cooperative puzzle platformer game built using Unity3D\cite{Unity3D} engine. 
The game involves two players at two distinct locations connected over the Internet. 
%During the game play, two players who are physically separated in different locations connect and cooperate to solve the puzzles in the game. 
The players cannot talk to each other directly and the only way to communicate is using their body language. 
Each player uses a Wii\cite{Wii} controller to move the avatar (e.g. left, right, jump) and a Kinect to capture their body language. 
%The wii controller is used to control the avator movement, like jumps and moves. 
%In the mean time, Kinect is applying the player's body language to the player's own avatar. 

%why use wii, why not use pure Kinect control!?



\section{Using Body Language}
% =============================================================================
%關卡設計方式
%In order to apply a new communication interface for user, Mute Robot is designed by using body language to communicate between players. Testing shows that using body language is an intuitive choice for player to communicate without speech ability.

To encourage players to use body language, we have designed an asymmetric puzzle system, with only one player receiving puzzle hints. The player will use body language to guide the other player to solve the puzzle.
Taking one of our game stage as an example (Figure 1), there is a locked door on the right side which obstructs both players' route to the next stage. 
%With a view to opening the locked door, the underside player will receive some puzzle-solving message. 
The top player can't see the puzzle-solving hints but can turn the wheel to the match the puzzle answer, which consequently opens the locked door.
The only way to pass the stage is for the bottom player to convey the puzzle-solving messages to the other player with body language. 
%As a result, the right side locked door will be opened when all wheels are turned to the correct direction.

%If human beings lose speech ability, with no doubt, an intuition way for people communication is using body language. Based on past historical experience, using body language is a human instinct. In addition,  We assume that 

% =============================================================================
\section{Testing and Evaluation}
% =============================================================================

We recruited 16 participants, and randomly divided them into 8 groups. In each group, participants did not know each other. We captured the players' movement through a video camera and the gameplay was also screen-recorded by Fraps\cite{Fraps}.
On average, the total game duration was about 10 to 20 minutes. 
At the end of the game, we provided a five-point Likert scale questionnaire to 
collect players' feedback to improve the gameplay experience.
%investigate the gameplay experience and collect player feedback.

\marginpar{
\begin{figure}
  \centering
  \includegraphics[width=0.8\linewidth]{figures/Figure3.jpg}
  \caption{A participant performing pictogram (letter ``N'') with body language.}
  \label{fig:Figure3}
\end{figure}
}

\begin{figure}
  \centering
  \includegraphics[width=0.8\linewidth]{figures/Figure2.jpg}
  \caption{A sequence puzzle from Mute Robot. The top player knows the correct sequence and is showing the bottom player to step on the center yellow button among the three buttons.}
  \label{fig:Figure2}
\end{figure}

\subsection{Communication Patterns}

 Our recorded video reveals some patterns on how the players communicate. We summarize these patterns below:
%According to the recorded video, we found some unique strategies between players' body language communication: 

(1) {\bf Repeat after me}: player who received puzzle-solving hints would perform all the puzzle-solving actions in one go for the other player to observe and emulate. For example, in one of our game stages (Figure. 3), the 3 buttons on the floor had to be stepped on one after another in a specific order. The top player would perform the answer all at once for the underside player to repeat.

(2) {\bf Step-by-Step}: player who received puzzle-solving hints would command the other player to do one action at a time. The next command would not be given until the previous command was executed correctly. For instance, a player jumped in place several times in order to imply that the other player should stand on the object at the corresponding position.

(3) {\bf Pictogram}: players would use their own body to express and mimic the hints. As shown in Figure 2, one participant wanted to express the letter ``N'' to the other player. Her solution was using her body to perform pictogram to show the character.

%AAAA
We observed that the {\emph Step-by-step style} is most used across player groups and game stages. 
%However, when faced with the floor buttons stage mentioned above, players would spontaneously adopt the {\emph repeat after me} style because the answer to the stage is a continuous series of actions.
In spite of large diversity of body language communication, the players can still find way to communicate effectively.
%Command style最多
%但也會跟關卡有所關係, ex: 踩踏的關卡有repeat after me 明顯表現,但轉轉輪救沒有


%
\subsection{Results from Questionnaires}

The results from the questionnaires show that our Mute Robot game design is practical and interesting. 
Figure 4 shows that all participants thought it was interesting to use body language to communicate.
%有趣,了解對方的語言、有好感
Figure 5 shows that about two-thirds of the players reported they could understand the other player's body language well.

\marginpar{
\begin{figure}
  \centering
  \includegraphics[width=1\linewidth]{figures/1_BLisInteresting.png}
  \caption{Participant rating of "Using body language is interesting".}
  \label{fig:1_BLisInteresting}
\end{figure}
}

\marginpar{
\begin{figure}
  \centering
  \includegraphics[width=1\linewidth]{figures/2_BLunderstand.png}
  \caption{Participant rating of "Understood the partner's body language well".}
  \label{fig:2_BLunderstand}
\end{figure}
}
%\begin{figure}
%  \centering
%  \includegraphics[width=0.8\linewidth]{figures/3_BLPositiveFeeling.png}
%  \caption{333}
%  \label{fig:3_BLPositiveFeeling}
%\end{figure}

%situations
%1. 命令句:如果畫面鐘有那個東西,會直接用指的, 回來手勢,
%2. 做給他看(踩左右中)
%3. 象行(angry中的N), 模仿動物 directly

%找16位玩家,不認識對方
%遊戲長度約10~20分鐘
%- # 
%- Body Language Communication (observation)
%- Satisaction Survey



% =============================================================================
\section{Future Work}
% =============================================================================

Currently, our system using Kinect (v1) only tracks significant body movement, which leaves behind important cues in facial expression and finger movement. 
%As far as we are concerned, with these detailed information, we can communicate with body language more precisely. 
For example, players can express their emotion through facial expression such as frowning and smiles, which may help players sense their teammate's feelings and connect them more deeply. On the other hand, players may use their fingers to express numbers and letters too easily, making their body language less entertaining. 
%加例子!
%有手指跟表情,溝通更容易&直觀

%We believe with these detailed information, we can communicate with body language more precisely. 
%Kinect allows personal computer to track the gesture of our body as we move around, but it hasn't had the fine detail of our gestures, and particular, it hasn't really understood our hand. If one day, the ability for Kinect can recognize our hand gesture and facial expression, we can communicate with body language more precisely. 

Futhermore, Mute Robot applies asymmetric puzzle system in which
only one-way message passing is required to pass all the stages.
%players cooperate with each other. In other words, it only requires one-way message passing.
%one way
%若有雙向溝通感情會更好,因兩人的參與度可以提升 or 成就感可以便高(都有做到事情....)
We plan to explore stages that supports two-way message passing between players, like letting both players receive part of puzzle-solving hints and exchange messages with body language.
We believe that it will promote even more interaction between players and make the stages more challenging.

%1. 可以抓到表情、手指
%2. 可以做更複雜的關卡
%now:一個人告訴另一個人,不是兩邊整合訊息 (a告訴b
%improve:上下皆有一些資料,共同完成 a purpose

% =============================================================================
\section{Conclusion}
% =============================================================================
We have designed Mute Robot to transcend language barriers and connect people via cooperative gameplay and communication through body language.
%加合作
%The Kinect-based 
% users to be more expressive than 
%Our design can facilitate the body language interaction and intrinsically motivate the player.
%The interaction between game and human body movement form an unprecedented innovative gameplay experiment.
%Mute Robot can proliferate cooperation and interaction between players.
Our user study results indicate that body language-based cooperative gameplay is practical and entertaining.
We hope to inspire more exploration of using body language in game designs.

%加一句真正可以做什麼
%Mute Robot provides a path towards future exploration of this relatively unexplored field of game design with capacious development space.

%數據 feadback 喜歡 
%這樣的作法、遊戲是蠻有發展空間的
%是一個很適合在kinect上的發展遊戲


\section{Acknowledgements}
We thank our advisor Prof. Mike Y. Chen and the faculty and staff of National Taiwan University.
We would also like to express our gratitude towards all players and testers who have helped us in our many (buggy) iterations. 


%\section{References format}
%References must be the same font size as other body text.
% REFERENCES FORMAT
% References must be the same font size as other body text.

\balance
\bibliographystyle{acm-sigchi}
\bibliography{chi14SGC-MuteRobot}

\end{document}