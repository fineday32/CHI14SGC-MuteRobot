\documentclass{chi-ext}
% Please be sure that you have the dependencies (i.e., additional LaTeX packages) to compile this example.
% See http://personales.upv.es/luileito/chiext/

%% EXAMPLE BEGIN -- HOW TO OVERRIDE THE DEFAULT COPYRIGHT STRIP -- (July 22, 2013 - Paul Baumann)
% \copyrightinfo{Permission to make digital or hard copies of all or part of this work for personal or classroom use is granted without fee provided that copies are not made or distributed for profit or commercial advantage and that copies bear this notice and the full citation on the first page. Copyrights for components of this work owned by others than ACM must be honored. Abstracting with credit is permitted. To copy otherwise, or republish, to post on servers or to redistribute to lists, requires prior specific permission and/or a fee. Request permissions from permissions@acm.org. \\
% {\emph{CHI'14}}, April 26--May 1, 2014, Toronto, Canada. \\
% Copyright \copyright~2014 ACM ISBN/14/04...\$15.00. \\
% DOI string from ACM form confirmation}
%% EXAMPLE END -- HOW TO OVERRIDE THE DEFAULT COPYRIGHT STRIP -- (July 22, 2013 - Paul Baumann)

%\title{CHI \LaTeX\ Ext. Abstracts Template}
\title{Mute Robot - Applying Cooperative Platformer with Body Language Communication}

\numberofauthors{6}
% Notice how author names are alternately typesetted to appear ordered in 2-column format;
% i.e., the first 4 autors on the first column and the other 4 auhors on the second column.
% Actually, it's up to you to strictly adhere to this author notation.
\author{
  \alignauthor{
  	\textbf{First Author}\\
  	\affaddr{AuthorCo, Inc.}\\
  	\affaddr{123 Author Ave.}\\
  	\affaddr{Authortown, PA 54321 USA}\\
  	\email{author1@anotherco.com}
  }\alignauthor{
  	\textbf{Fourth Author}\\
  	\affaddr{AuthorCo, Inc.}\\
  	\affaddr{123 Author Ave.}\\
  	\affaddr{Authortown, PA 54321 USA}\\
  	\email{author5@anotherco.com}
  }
  \vfil
  \alignauthor{
  	\textbf{Second Author}\\
  	\affaddr{AuthorCo, Inc.}\\
  	\affaddr{123 Author Ave.}\\
  	\affaddr{Authortown, PA 54321 USA}\\
  	\email{author2@anotherco.com}
  }\alignauthor{
  	\textbf{Fifth Author}\\
  	\affaddr{AuthorCo, Inc.}\\
  	\affaddr{123 Author Ave.}\\
  	\affaddr{Authortown, PA 54321 USA}\\
  	\email{author6@anotherco.com}
  }
  \vfil
  \alignauthor{
  	\textbf{Third Author}\\
  	\affaddr{AuthorCo, Inc.}\\
  	\affaddr{123 Author Ave.}\\
  	\affaddr{Authortown, PA 54321 USA}\\
  	\email{author3@anotherco.com}
  }\alignauthor{
  	\textbf{Sixth Author}\\
  	\affaddr{AuthorCo, Inc.}\\
  	\affaddr{123 Author Ave.}\\
  	\affaddr{Authortown, PA 54321 USA}\\
  	\email{author7@anotherco.com}
  }
}

% Paper metadata (use plain text, for PDF inclusion and later re-using, if desired)
\def\plaintitle{CHI LaTeX Extended Abstracts Template}
\def\plainauthor{Luis A. Leiva}
\def\plainkeywords{Guides, instructions, author's kit, conference publications}
\def\plaingeneralterms{Documentation, Standardization}

\hypersetup{
  % Your metadata go here
  pdftitle={\plaintitle},
  pdfauthor={\plainauthor},  
  pdfkeywords={\plainkeywords},
  pdfsubject={\plaingeneralterms},
  % Quick access to color overriding:
  %citecolor=black,
  %linkcolor=black,
  %menucolor=black,
  %urlcolor=black,
}

\usepackage{graphicx}   % for EPS use the graphics package instead
\usepackage{balance}    % useful for balancing the last columns
\usepackage{bibspacing} % save vertical space in references


\begin{document}

\maketitle

\begin{abstract}
Speech ability, a general habit becoming nature for us, plays an important role in our daily life, but if one day we can't speak anymore, using body language becomes a evident feasible way to communicate between human beings.
Mute Robot Game has been designed to provide players a new environment where they become a robot without speech ability. The game intend to supply players facing in challenging stages. 
Players communicate with body language, with an emphasis on cooperation to facilitate immersion with each other.
%Players need to cooperate with each other to pass challenges by using body language communication.
\end{abstract}

%\begin{abstract}
%In this sample we describe the formatting requirements for various %SIGCHI related submissions 
%and offer recommendations on writing for the worldwide SIGCHI %readership. 
%%Do not change the page size or page settings.
%Please review this document even if you have submitted to SIGCHI %conferences before, 
%some format details have changed relative to previous years.
%\end{abstract}

%[Section] Author Keywords
\keywords{\plainkeywords}
\textcolor{red}{Mandatory section to be included in your final version.}

%[Section] ACM Classification Keywords
\category{H.5.m}{Information interfaces and presentation (e.g., HCI)}{Miscellaneous}. 
%See \cite{ACMCCS} 
See: \url{http://www.acm.org/about/class/1998/} 
for help using the ACM Classification system.
\textcolor{red}{Mandatory section to be included in your final version.}

%[Section] General Terms
\terms{\plaingeneralterms}
\textcolor{red}{Optional section to be included in your final version.}

% =============================================================================
\section{Introduction}
% =============================================================================
Using body langauge to communicate with people from different regions or speaking different languages is a common solution. 
Therefore, we try to construct a creative gameplay to immerse this communication way in the reality to a virtual world. 
Although Ways[] proposed a similar gameplay in 2011, they still used keyboards and mouses as input devices.
Because of the restriction of controlling avatars in Ways[] by keyboards and mouses, we provides Mute Robot, a novel game combining kinect to capture players postures and then precisely apply them to avatars in the game.
Mute Robot is a cooperative platformer with two players connected over the internet.
By corresponding players' motions to avatars in the game, Mute Robot attempts to create an innovative gameplay of letting players communicate with each other only using body language.
Thus, it is a cross-language game for humans around the world.

Without any other ways of communication, players need to operate on the same wavelength or build their own connection pattern by body language at the beginning.
After finishing building their own connection between each other, players start to solve a serial of puzzles with their own communication pattern.
For example, in the first level, there is a door locked on the right side, players have to tread three buttons on the ground with the correct sequence to unlock the door.(fig. 2)
However, only the upper side player know the sequence, the underside player can trample those buttons, so they need to pass the right pattern with the communication they built before.

The focus in Mute Robot lies on this cross-language desgin for the core gameplay.
In order to make all players experience both acting an information sender and a receiver, we switch password stages between the upper side and the down side at different levels.
Morevoer, Mute Robot desires to show the power of body language, so the game require the password sender acting more and more difficult objects with body language.
For example, in the second stage, there are a circle and a x on the first wheel and then three numbers on the second one. It increases the difficulty on the communication between the two players.(fig. 1)
Thus, players need to evaluate their own connection stronger and stronger to deal with the increasing difficulty in the game. 

 
  
% =============================================================================
\section{Design and Development}
% =============================================================================
line

% =============================================================================
\section{Future Work}
% =============================================================================
line


% =============================================================================
\section{Conclusion}
% =============================================================================
Mute Robot shows a new integration between current gameplay style and body language, where the interaction between game and human body movement form a unprecedented innovative gameplay empiriment.
Our statistics results indicate that Mute Robot can proliferation of communication and interaction between players.
The Kinect-based manipulation can facilitate the body language interation and intrinsically motivate the player.
Mute Robot provide a path towards future exploration of this relatively unexplored field of game design with capacious development space.

%數據 feadback 喜歡 
%這樣的作法、遊戲是蠻有發展空間的
%是一個很適合在kinect上的發展遊戲


\section{Acknowledgements}
We thank advisor Mike Y. Chen and the faculty and staff of National Taiwan University.
We would also like to express our gratitude towards all playtesters who have helped us in our many iterations. 


\section{References format}
References must be the same font size as other body text.
% REFERENCES FORMAT
% References must be the same font size as other body text.

\balance
\bibliographystyle{acm-sigchi}
\bibliography{sample}

\end{document}